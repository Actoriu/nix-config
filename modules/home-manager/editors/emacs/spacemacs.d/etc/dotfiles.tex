% Created 2021-09-14 二 16:56
% Intended LaTeX compiler: xelatex
\documentclass[11pt]{article}
\usepackage[utf8]{inputenc}
\usepackage[T1]{fontenc}
\usepackage{fixltx2e}
\usepackage{wrapfig}
\usepackage{soul}
\usepackage{textcomp}
\usepackage{marvosym}
\usepackage{wasysym}
\usepackage{latexsym}
\usepackage{amssymb}
\usepackage{hyperref}
\usepackage[Lenny]{fncychap}
\usepackage[figuresright]{rotating}
\usepackage{capt-of}
\usepackage{amssymb}
\usepackage[normalem]{ulem}
\usepackage{wrapfig}
\usepackage{grffile}
\usepackage{booktabs}
\usepackage{tabularx}
\usepackage{amsmath}
\usepackage{textcomp}
\usepackage{fancyhdr}
\usepackage{tikz}
\usepackage{longtable}
\usepackage{float}
\usepackage{geometry}
\usepackage{xunicode}
\usepackage{indentfirst}
\usepackage{fontspec}
\usepackage{listings}
\usepackage{xcolor}
\usepackage{graphicx}
\usepackage{longtable}
\usepackage{float}
\date{\today}
\title{configuration}
\hypersetup{
 pdfauthor={},
 pdftitle={configuration},
 pdfkeywords={},
 pdfsubject={},
 pdfcreator={Emacs 27.2 (Org mode 9.4.4)}, 
 pdflang={Zh-Cn}}
\begin{document}

\tableofcontents


\section{\texttt{X11}}
\label{sec:org537615a}

\subsection{\texttt{xinitrc}}
\label{sec:org7cc1509}

\lstset{language=bash,label= ,caption= ,captionpos=b,numbers=none}
\begin{lstlisting}
#
# ~/.xinitrc
#
# Executed by startx (run your window manager from here)

if [ -d /etc/X11/xinit/xinitrc.d ]; then
    for f in /etc/X11/xinit/xinitrc.d/?*.sh ; do
        [ -x "$f" ] && . "$f"
    done
    unset f
fi

# merge in defaults and keymaps
# custom x resources
if [ -s $HOME/.Xresources ]; then
    xrdb -merge "$HOME/.Xresources"
fi

if [ -s $HOME/.Xmodmap ]; then
    xmodmap "$HOME/.Xmodmap"
fi

# Normal cursor
xsetroot -cursor_name left_ptr &

# Fcitx
export GTK_IM_MODULE=fcitx
export QT_IM_MODULE=fcitx
export XMODIFIERS=@im=fcitx

# Locale
# export LANG=zh_CN.UTF-8
# export LANGUAGE=zh_CN:en_US
# export LC_CTYPE=en_US.UTF-8

# Window Manager
# Awesome is kept as default
session=${1:-awesome}

case $session in
    awesome           ) exec awesome >> ~/.cache/awesome/stdout 2>> ~/.cache/awesome/stderr;;
    bspwm             ) exec bspwm;;
    catwm             ) exec catwm;;
    sawfish           ) exec sawfish;;
    spectrwm          ) exec spectrwm;;
    cinnamon          ) exec cinnamon-session;;
    dwm               ) exec dwm;;
    enlightenment     ) exec enlightenment_start;;
    ede               ) exec startede;;
    fluxbox           ) exec startfluxbox;;
    gnome             ) exec gnome-session;;
    gnome-classic     ) exec gnome-session --session=gnome-classic;;
    i3|i3wm           ) exec i3;;
    icewm             ) exec icewm-session;;
    jwm               ) exec jwm;;
    kde               ) exec startkde;;
    mate              ) exec mate-session;;
    monster|monsterwm ) exec monsterwm;;
    notion            ) exec notion;;
    openbox           ) exec openbox-session;;
    unity             ) exec unity;;
    xfce|xfce4        ) exec startxfce4;;
    xmonad            ) exec xmonad;;
    # No known session, try to run it as command
    *                 ) exec $1;;
esac
\end{lstlisting}

\subsection{\texttt{xprofile}}
\label{sec:org042f4c9}

\lstset{language=bash,label= ,caption= ,captionpos=b,numbers=none}
\begin{lstlisting}
#
# ~/.xprofile
#

# merge in defaults and keymaps
# custom x resources
if [ -s $HOME/.Xresources ]; then
    xrdb -merge "$HOME/.Xresources"
fi

if [ -s $HOME/.Xmodmap ]; then
    xmodmap "$HOME/.Xmodmap"
fi

# Fcitx
export GTK_IM_MODULE=fcitx
export QT_IM_MODULE=fcitx
export XMODIFIERS=@im=fcitx

# Locale
# export LANG=zh_CN.UTF-8
# export LANGUAGE=zh_CN:en_US
# export LC_CTYPE=en_US.UTF-8
\end{lstlisting}

\subsection{\texttt{path}}
\label{sec:orgaa49e96}

\subsubsection{\texttt{print colors}}
\label{sec:org6f43969}

\lstset{language=bash,label= ,caption= ,captionpos=b,numbers=none}
\begin{lstlisting}
#
#   This file echoes a bunch of color codes to the
#   terminal to demonstrate what's available.  Each
#   line is the color code of one forground color,
#   out of 17 (default + 16 escapes), followed by a
#   test use of that color on all nine background
#   colors (default + 8 escapes).
#

T='gYw'   # The test text

echo -e "\n                 40m     41m     42m     43m\
     44m     45m     46m     47m";

for FGs in '    m' '   1m' '  30m' '1;30m' '  31m' '1;31m' '  32m' \
                   '1;32m' '  33m' '1;33m' '  34m' '1;34m' '  35m' '1;35m' \
                   '  36m' '1;36m' '  37m' '1;37m';
do FG=${FGs// /}
   echo -en " $FGs \033[$FG  $T  "
   for BG in 40m 41m 42m 43m 44m 45m 46m 47m;
   do echo -en "$EINS \033[$FG\033[$BG  $T  \033[0m";
   done
   echo;
done
echo
\end{lstlisting}

\subsubsection{\texttt{tmux script}}
\label{sec:orgc644b87}

\href{https://github.com/xuxiaodong/tmuxen}{tmux script from xu xiaodong}

\lstset{language=bash,label= ,caption= ,captionpos=b,numbers=none}
\begin{lstlisting}
#
# name     : tmuxen, tmux environment made easy
# author   : xxx <xxx@xxx.com>
# license  : GPL3
# created  : 2012 Jul 01
# modified : 2013 Jul 06
#

cmd=$(which tmux)      # tmux path
session=$(hostname -s) # session name

if [ -z $cmd ]; then
    echo "You need to install tmux."
    exit 1
fi

$cmd has -t $session 2> /dev/null

if [ $? != 0 ]; then
    $cmd new -d -n vim -s $session "vim"
    #$cmd neww -n mutt -t $session "mutt"
    #$cmd neww -n irssi -t $session "irssi"
    #$cmd neww -n cmus -t $session "cmus"
    $cmd splitw -v -p 30 -t $session "zsh"
    $cmd selectw -t $session:1
fi

$cmd att -t $session

exit 0
\end{lstlisting}

\subsection{\texttt{xresources}}
\label{sec:org680d6d4}

\subsubsection{\texttt{fonts}}
\label{sec:org4396fa9}

\lstset{language=conf-xdefaults,label= ,caption= ,captionpos=b,numbers=none}
\begin{lstlisting}
! -- 字体 -- !
Xft*dpi                               : 106
Xft*antialias                         : True
Xft*autohint                          : False
Xft*rgba                              : rgb
Xft*lcdfilter                         : lcdlight
Xft*hinting                           : True
Xft*hintstyle                         : hintslight
\end{lstlisting}

\lstset{language=conf-xdefaults,label= ,caption= ,captionpos=b,numbers=none}
\begin{lstlisting}
URxvt*font:                           xft:DejaVuSansMonoForPowerline Nerd Font:regular:size=16,xft:WenQuanYi Micro Hei Mono:regular:size=16
URxvt*boldFont:                       xft:DejaVuSansMonoForPowerline Nerd Font:bold:size=16,xft:WenQuanYi Micro Hei Mono:bold:size=16
URxvt*italicFont:                     xft:DejaVuSansMonoForPowerline Nerd Font:italic:size=16,xft:WenQuanYi Micro Hei Mono:italic:size=16
URxvt*boldItalicFont:                 xft:DejaVuSansMonoForPowerline Nerd Font:bold:italic:size=16,xft:WenQuanYi Micro Hei Mono:bold:italic:size=16
\end{lstlisting}

\subsubsection{\texttt{colors}}
\label{sec:orgd015990}

\href{https://github.com/morhetz/gruvbox-generalized}{gruvbox-generalized}

\lstset{language=conf-xdefaults,label= ,caption= ,captionpos=b,numbers=none}
\begin{lstlisting}
! -----------------------------------------------------------------------------
! File: gruvbox-dark.xresources
! Description: Retro groove colorscheme generalized
! Author: morhetz <morhetz@gmail.com>
! Source: https://github.com/morhetz/gruvbox-generalized
! Last Modified: 6 Sep 2014
! -----------------------------------------------------------------------------

! special
*background:    #282828
*foreground:    #ebdbb2
*.colorUL:      #7F9F7F
*.colorBD:      #FF1493
*.cursorColor:  #7F9F7F
! *.pointerColorBackground:         #586E75
! *.pointerColorForeground:         #93A1A1

! Black + DarkGrey
*color0:        #282828
*color8:        #928374

! DarkRed + Red
*color1:        #cc241d
*color9:        #fb4934

! DarkGreen + Green
*color2:        #98971a
*color10:       #b8bb26

! DarkYellow + Yellow
*color3:        #d79921
*color11:       #fabd2f

! DarkBlue + Blue
*color4:        #458588
*color12:       #83a598

! DarkMagenta + Magenta
*color5:        #b16286
*color13:       #d3869b

! DarkCyan + Cyan
*color6:        #689d6a
*color14:       #8ec07c

! LightGrey + White
*color7:        #a89984
*color15:       #ebdbb2
\end{lstlisting}

\lstset{language=conf-xdefaults,label= ,caption= ,captionpos=b,numbers=none}
\begin{lstlisting}
! -----------------------------------------------------------------------------
! File: gruvbox-urxvt256.xresources
! Description: Retro groove colorscheme generalized
! Author: morhetz <morhetz@gmail.com>
! Source: https://github.com/morhetz/gruvbox-generalized
! Last Modified: 13 Dec 2013
! -----------------------------------------------------------------------------

URxvt.color24:  #076678
URxvt.color66:  #427b58
URxvt.color88:  #9d0006
URxvt.color96:  #8f3f71
URxvt.color100: #79740e
URxvt.color108: #8ec07c
URxvt.color109: #83a598
URxvt.color130: #af3a03
URxvt.color136: #b57614
URxvt.color142: #b8bb26
URxvt.color167: #fb4934
URxvt.color175: #d3869b
URxvt.color208: #fe8019
URxvt.color214: #fabd2f
URxvt.color223: #ebdbb2
URxvt.color228: #f2e5bc
URxvt.color229: #fbf1c7
URxvt.color230: #f9f5d7
URxvt.color234: #1d2021
URxvt.color235: #282828
URxvt.color236: #32302f
URxvt.color237: #3c3836
URxvt.color239: #504945
URxvt.color241: #665c54
URxvt.color243: #7c6f64
URxvt.color244: #928374
URxvt.color245: #928374
URxvt.color246: #a89984
URxvt.color248: #bdae93
URxvt.color250: #d5c4a1
\end{lstlisting}

\subsubsection{\texttt{config}}
\label{sec:orge1c6858}

\begin{enumerate}
\item \texttt{urxvt}
\label{sec:org44c3a32}

install urxvt-perls package

\lstset{language=conf-xdefaults,label= ,caption= ,captionpos=b,numbers=none}
\begin{lstlisting}
! -- 颜色 -- !
URxvt*depth                           : 32

! -- 渲染 -- !
URxvt*preeditType                     : Root
URxvt*skipBuiltinGlyphs               : True

! -- 伪透明 -- !
URxvt*transparent                     : False
! -- 透明度 -- !
URxvt*shading                         : 50

! -- 标题 -- !
URxvt*termName                        : xterm-256color
URxvt*title                           : Urxvt
!URxvt*backgroundPixmap:
!URxvt*iconFile:

! -- 大小/位置 -- !
URxvt*geometry                        : 80x20

! -- 光标字符选择 -- !
URxvt*cutchars                        : "\\ `\"\'&()*,;<_=>?%#@~[-A-Za-z0-9]{|}.:+-"

! -- Ctrl+Shift的输入特殊字符功能 -- !
URxvt*iso14755                        : False
URxvt*iso14755_52                     : False

! -- 滚动条 -- !
URxvt*scrollBar                       : False
URxvt*scrollBar_right                 : True
URxvt*scrollstyle                     : rxvt
URxvt*scrollTtyOutput                 : False
URxvt*scrollTtyKeypress               : True
URxvt*scrollWithBuffer                : True

! -- 滚屏 -- !
URxvt*mouseWheelScrollPage            : True
URxvt*saveLines                       : 65535
URxvt*secondaryScreen                 : True
URxvt*secondaryScroll                 : True
URxvt.secondaryWheel                  : True

! -- 光标 -- !
URxvt*cursorBlink                     : True
URxvt*cursorUnderline                 : True

! -- 边框 -- !
URxvt*borderLess                      : False

! -- 扩展 -- !
URxvt*perl-lib                        : /usr/lib/urxvt/perl/
URxvt*perl-ext-common                 : default,clipboard,matcher,keyboard-select,url-select
! -- clipboard -- !
URxvt*keysym.M-w                      : perl:clipboard:copy
URxvt*keysym.C-y                      : perl:clipboard:paste
URxvt.keysym.M-g                      : perl:clipboard:paste_escaped
! -- keyboard-select -- !
URxvt*keysym.M-Escape                 : perl:keyboard-select:activate
URxvt*keysym.M-s                      : perl:keyboard-select:search
! -- url-select -- !
URxvt*keysym.M-u                      : perl:url-select:select_next

! -- 复制 -- !
URxvt*clipboard.autocopy              : True

! -- 超链接 -- !
URxvt*url-select.autocopy             : True
URxvt*url-select.launcher             : /usr/bin/xdg-open
URxvt*url-select.underline            : True
\end{lstlisting}

\item \texttt{xterm}
\label{sec:org562ea2f}

\lstset{language=conf-xdefaults,label= ,caption= ,captionpos=b,numbers=none}
\begin{lstlisting}
! Application Resources
xterm.locale                          : True
xterm.termName                        : xterm-256color
xterm.utmpInhibit                     : True

! VT100 Widget Resources
xterm*allowTitleOps                   : False
xterm*altSendsEscape                  : True
xterm*bellIsUrgent                    : True
xterm*borderLess                      : True
! allow selecting email/url by double click
xterm*charClass                       : 33:48,37-38:48,45-47:48,64:48,58:48,126:48,61:48,63:48,43:48,35:48
xterm*dynamicColors                   : True
xterm*colorBDMode                     : True
xterm*colorULMode                     : True
xterm*cursorBlink                     : True
xterm*cursorUnderLine                 : True
xterm*eightBitInput                   : False
! uncomment to output a summary of each font s metrics
! xterm.reportFonts                     : true
xterm*fontMenu*fontdefault*Label      :Default
xterm*faceName                        : DejaVuSansMonoForPowerline Nerd Font:antialias=True:pixelsize=20
xterm*faceNameDoublesize              : WenQuanYi Micro Hei Mono:antialias=True:pixelsize=20
xterm*fastScroll                      : True
xterm*highlightSelection              : True
XTerm*jumpScroll                      : True
xterm*loginshell                      : True
XTerm*multiScroll                     : True
xterm*printAttributes                 : 2
xterm*printerCommand                  : xterm -T History -e sh -c 'less -r -o /tmp/xterm.dump <&3' 3<&0
xterm*rightScrollBar                  : True
xterm*saveLines                       : 65535
xterm*scrollBar                       : False
xterm*trimSelection                   : True
xterm*veryBoldColors                  : 4
xterm*xftAntialias                    : True

xterm*VT100.Translations: #override \
    Ctrl Shift <Key>C                 : copy-selection(CLIPBOARD) \n
    Ctrl Shift <Key>V                 : insert-selection(CLIPBOARD) \n
    Ctrl <Btn1Up>                     : exec-formatted("xdg-open '%t'", PRIMARY)
\end{lstlisting}

\lstset{language=conf-xdefaults,label= ,caption= ,captionpos=b,numbers=none}
\begin{lstlisting}
!-- colorscheme & font --!
#include                              ".Xresources.d/colors/gruvbox_urxvt256"
#include                              ".Xresources.d/colors/gruvbox_dark"
#include                              ".Xresources.d/fonts/Dejavu_wqy"
#include                              ".Xresources.d/fonts/fonts"
#include                              ".Xresources.d/urxvt"
#include                              ".Xresources.d/xterm"
\end{lstlisting}
\end{enumerate}

\section{\texttt{Tools}}
\label{sec:orgabaf4f3}

\subsection{\texttt{zsh}}
\label{sec:org937282a}

\subsubsection{\texttt{.zshrc}}
\label{sec:org8355c88}

\begin{enumerate}
\item \texttt{antigen}
\label{sec:orgceaccb7}

\lstset{language=bash,label= ,caption= ,captionpos=b,numbers=none}
\begin{lstlisting}
#
# ~/.zshrc
#

if [ ! -d $HOME/.antigen ]; then
    git clone https://github.com/zsh-users/antigen.git "${HOME}/.antigen"
fi

if [ -s $HOME/.antigen/antigen.zsh ]; then
    source "${HOME}/.antigen/antigen.zsh"
fi

# Load the oh-my-zsh's library.
antigen use oh-my-zsh

antigen bundles <<EOBUNDLES
  # Bundles from the default repo.
  colored-man-pages
  command-not-found
  git

  # Syntax highlighting bundle.
  zsh-users/zsh-completions
  # zsh-users/zsh-syntax-highlighting
  zdharma/fast-syntax-highlighting
  joel-porquet/zsh-dircolors-solarized
EOBUNDLES

# Load the theme.
antigen theme kidonchu/antigen-theme antigen

# Tell antigen that you're done.
antigen apply
\end{lstlisting}

\item \texttt{dircolor}
\label{sec:org8d47b7d}

\href{https://github.com/joel-porquet/zsh-dircolors-solarized}{zsh-dircolors-solarized}

\lstset{language=bash,label= ,caption= ,captionpos=b,numbers=none}
\begin{lstlisting}
dircolors.ansi-dark
\end{lstlisting}
\end{enumerate}

\subsubsection{\texttt{.zshenv}}
\label{sec:org87b132d}
\begin{enumerate}
\item \texttt{path}
\label{sec:org123c440}

\lstset{language=bash,label= ,caption= ,captionpos=b,numbers=none}
\begin{lstlisting}
#
# Defines runtime environment
#

# PATH
if [ -d $HOME/.bin ]; then
    PATH="$HOME/.bin:$PATH"
fi

if [ -d $HOME/.cargo/bin ]; then
    PATH="$HOME/.cargo/bin:$PATH"
fi
typeset -U PATH
export PATH
\end{lstlisting}

\item \texttt{houdini}
\label{sec:org3704350}

\lstset{language=bash,label= ,caption= ,captionpos=b,numbers=none}
\begin{lstlisting}
if [ -d /opt/hfs16.0.633 ]; then
    pushd /opt/hfs16.0.633 >/dev/null 2>&1; source houdini_setup >/dev/null 2>&1; popd >/dev/null 2>&1
fi
\end{lstlisting}

\item \texttt{exprot(global/rust)}
\label{sec:org3bc465b}

\lstset{language=bash,label= ,caption= ,captionpos=b,numbers=none}
\begin{lstlisting}
export GTAGSLABEL=pygments
export RUSTUP_DIST_SERVER=https://mirrors.ustc.edu.cn/rust-static
export RUSTUP_UPDATE_ROOT=https://mirrors.ustc.edu.cn/rust-static/rustup
export RUST_SRC_PATH="$(rustc --print sysroot)/lib/rustlib/src/rust/src"
\end{lstlisting}

\item \texttt{proxy}
\label{sec:orgc9c3c2a}

\href{https://github.com/acgtyrant/dotfiles}{proxy from acgtyrant}

\lstset{language=bash,label= ,caption= ,captionpos=b,numbers=none}
\begin{lstlisting}
# These addresses are assigned by cow.
proxy () {
    export http_proxy="http://127.0.0.1:7777"
    export https_proxy="http://127.0.0.1:7777"
    export HTTP_PROXY="http://127.0.0.1:7777"
    export HTTPS_PROXY="http://127.0.0.1:7777"
    echo "http proxy on"
}

noproxy () {
    unset http_proxy
    unset https_proxy
    unset HTTP_PROXY
    unset HTTPS_PROXY
    echo "http proxy off"
}
\end{lstlisting}
\end{enumerate}

\subsubsection{\texttt{zprofile}}
\label{sec:org42acbb6}

\href{https://github.com/xuxiaodong/dotman}{.zprofile from xu xiaodong}

\lstset{language=bash,label= ,caption= ,captionpos=b,numbers=none}
\begin{lstlisting}
#
# author:    xxx <xxx@xxx.com>
# modified:  2015 May 09
#

#-- source --#

. $HOME/.zshrc

#-- auto start x --#

# if [[ -z "$DISPLAY" ]] && [[ $(tty) = /dev/tty1 ]]; then
#     startx
#     logout
# fi
\end{lstlisting}

\subsection{\texttt{global/ctags}}
\label{sec:org490f6b3}

\lstset{language=bash,label= ,caption= ,captionpos=b,numbers=none}
\begin{lstlisting}
if [ -s /usr/share/gtags/gtags.conf ]; then
    cp /usr/share/gtags/gtags.conf ~/.globalrc
fi
\end{lstlisting}

\lstset{language=bash,label= ,caption= ,captionpos=b,numbers=none}
\begin{lstlisting}
#
# Copyright (c) 1998, 1999, 2000, 2001, 2002, 2003, 2010, 2011, 2013,
# 2015, 2016, 2017
# Tama Communications Corporation
#
# This file is part of GNU GLOBAL.
#
# This file is free software; as a special exception the author gives
# unlimited permission to copy and/or distribute it, with or without
# modifications, as long as this notice is preserved.
#
# This program is distributed in the hope that it will be useful, but
# WITHOUT ANY WARRANTY, to the extent permitted by law; without even the
# implied warranty of MERCHANTABILITY or FITNESS FOR A PARTICULAR PURPOSE.
#
# *
# Configuration file for GNU GLOBAL source code tagging system.
#
# Basically, GLOBAL doesn't need this configuration file ('gtags.conf'),
# because it has default values in itself. If you have this file as
# '/etc/gtags.conf' or "$HOME/.globalrc" then GLOBAL overwrite the default
# values with values in the file.
# Configuration file is also necessary to use plug-in parsers.
#
# The format is similar to termcap(5). You can specify a target with
# GTAGSLABEL environment variable. Default target is 'default'.
#
# If you want to have default values for yourself, it is recommended to
# use the following method:
#
# default:\
    # :tc=default@~/.globalrc:\ <= includes default values from ~/.globalrc.
# :tc=native:
#
# Please refer to gtags.conf(5) for details.
#
default:\
    :tc=native:
native:\
    :tc=gtags:tc=htags:
user:\
    :tc=user-custom:tc=htags:
ctags:\
    :tc=exuberant-ctags:tc=htags:
new-ctags:\
    :tc=universal-ctags:tc=htags:
pygments:\
    :tc=pygments-parser:tc=htags:
#
# [How to merge two or more parsers?]
#
# Rule: The first matched langmap is adopted.
#
# ":tc=builtin-parser:tc=pygments-parser:" means:
# If built-in parser exists for the target, it is used.
# Else if pygments parser exists it is used.
#
native-pygments:\
    :tc=builtin-parser:tc=pygments-parser:htags:
#---------------------------------------------------------------------
# Configuration for gtags(1)
# See gtags(1).
#---------------------------------------------------------------------
common:\
    :skip=HTML/,HTML.pub/,tags,TAGS,ID,y.tab.c,y.tab.h,gtags.files,cscope.files,cscope.out,cscope.po.out,cscope.in.out,SCCS/,RCS/,CVS/,CVSROOT/,{arch}/,autom4te.cache/,*.orig,*.rej,*.bak,*~,#*#,*.swp,*.tmp,*_flymake.*,*_flymake,*.o,*.a,*.so,*.lo,*.zip,*.gz,*.bz2,*.xz,*.lzh,*.Z,*.tgz,*.min.js,*min.css:
#
# Built-in parsers.
#
gtags:\
    :tc=common:\
    :tc=builtin-parser:
builtin-parser:\
    :langmap=c\:.c.h,yacc\:.y,asm\:.s.S,java\:.java,cpp\:.c++.cc.hh.cpp.cxx.hxx.hpp.C.H,php\:.php.php3.phtml:
#
# skeleton for user's custom parser.
#
user-custom|User custom plugin parser:\
                 :tc=common:\
                 :langmap=c\:.c.h:\
                 :gtags_parser=c\:$libdir/gtags/user-custom.so:
#
# Plug-in parser to use Exuberant Ctags.
#
exuberant-ctags|plugin-example|setting to use Exuberant Ctags plug-in parser:\
                                       :tc=common:\
                                       :ctagscom=/usr/bin/ctags:\
                                       :ctagslib=$libdir/gtags/exuberant-ctags.so:\
                                       :tc=common-ctags-maps:
#
# A common map for both Exuberant Ctags and Universal Ctags.
# Don't include definitions of ctagscom and ctagslib in this entry.
#
common-ctags-maps:\
    # Ant      *.build.xml        (out of support)
    # Asm      *.[68][68][kKsSxX] *.[xX][68][68]  (out of support)
    :langmap=Asm\:.asm.ASM.s.S.A51.29k.29K:\
        :langmap=Asp\:.asp.asa:\
        :langmap=Awk\:.awk.gawk.mawk:\
        :langmap=Basic\:.bas.bi.bb.pb:\
        :langmap=BETA\:.bet:\
        :langmap=C\:.c:\
        :langmap=C++\:.c++.cc.cp.cpp.cxx.h.h++.hh.hp.hpp.hxx:\
        :langmap=C#\:.cs:\
        :langmap=Cobol\:.cbl.cob.CBL.COB:\
        :langmap=DosBatch\:.bat.cmd:\
        :langmap=Eiffel\:.e:\
        :langmap=Erlang\:.erl.ERL.hrl.HRL:\
        :langmap=Flex\:.as.mxml:\
        :langmap=Fortran\:.f.for.ftn.f77.f90.f95:\
        :langmap=HTML\:.htm.html:\
        :langmap=Java\:.java:\
        :langmap=JavaScript\:.js:\
        :langmap=Lisp\:.cl.clisp.el.l.lisp.lsp:\
        :langmap=Lua\:.lua:\
        # Make  [Mm]akefile GNUmakefile     (out of support)
        :langmap=Make\:.mak.mk:\
            :langmap=MatLab\:.m:\
            :langmap=OCaml\:.ml.mli:\
            :langmap=Pascal\:.p.pas:\
            :langmap=Perl\:.pl.pm.plx.perl:\
            :langmap=PHP\:.php.php3.phtml:\
            :langmap=Python\:.py.pyx.pxd.pxi.scons:\
            :langmap=REXX\:.cmd.rexx.rx:\
            :langmap=Ruby\:.rb.ruby:\
            :langmap=Scheme\:.SCM.SM.sch.scheme.scm.sm:\
            :langmap=Sh\:.sh.SH.bsh.bash.ksh.zsh:\
            :langmap=SLang\:.sl:\
            :langmap=SML\:.sml.sig:\
            :langmap=SQL\:.sql:\
            :langmap=Tcl\:.tcl.tk.wish.itcl:\
            :langmap=Tex\:.tex:\
            :langmap=Vera\:.vr.vri.vrh:\
            :langmap=Verilog\:.v:\
            :langmap=VHDL\:.vhdl.vhd:\
            :langmap=Vim\:.vim:\
            :langmap=YACC\:.y:\
            :gtags_parser=Asm\:$ctagslib:\
            :gtags_parser=Asp\:$ctagslib:\
            :gtags_parser=Awk\:$ctagslib:\
            :gtags_parser=Basic\:$ctagslib:\
            :gtags_parser=BETA\:$ctagslib:\
            :gtags_parser=C\:$ctagslib:\
            :gtags_parser=C++\:$ctagslib:\
            :gtags_parser=C#\:$ctagslib:\
            :gtags_parser=Cobol\:$ctagslib:\
            :gtags_parser=DosBatch\:$ctagslib:\
            :gtags_parser=Eiffel\:$ctagslib:\
            :gtags_parser=Erlang\:$ctagslib:\
            :gtags_parser=Flex\:$ctagslib:\
            :gtags_parser=Fortran\:$ctagslib:\
            :gtags_parser=HTML\:$ctagslib:\
            :gtags_parser=Java\:$ctagslib:\
            :gtags_parser=JavaScript\:$ctagslib:\
            :gtags_parser=Lisp\:$ctagslib:\
            :gtags_parser=Lua\:$ctagslib:\
            :gtags_parser=Make\:$ctagslib:\
            :gtags_parser=MatLab\:$ctagslib:\
            :gtags_parser=OCaml\:$ctagslib:\
            :gtags_parser=Pascal\:$ctagslib:\
            :gtags_parser=Perl\:$ctagslib:\
            :gtags_parser=PHP\:$ctagslib:\
            :gtags_parser=Python\:$ctagslib:\
            :gtags_parser=REXX\:$ctagslib:\
            :gtags_parser=Ruby\:$ctagslib:\
            :gtags_parser=Scheme\:$ctagslib:\
            :gtags_parser=Sh\:$ctagslib:\
            :gtags_parser=SLang\:$ctagslib:\
            :gtags_parser=SML\:$ctagslib:\
            :gtags_parser=SQL\:$ctagslib:\
            :gtags_parser=Tcl\:$ctagslib:\
            :gtags_parser=Tex\:$ctagslib:\
            :gtags_parser=Vera\:$ctagslib:\
            :gtags_parser=Verilog\:$ctagslib:\
            :gtags_parser=VHDL\:$ctagslib:\
            :gtags_parser=Vim\:$ctagslib:\
            :gtags_parser=YACC\:$ctagslib:
#
# Plug-in parser to use Universal Ctags.
#
universal-ctags|setting to use Universal Ctags plug-in parser:\
                        :tc=common:\
                        :ctagscom=:\
                        :ctagslib=$libdir/gtags/universal-ctags.so:\
                        :tc=common-ctags-maps:\
                        :langmap=Ada\:.adb.ads.Ada:\
                        # Please uncomment to use this entry.
                        # :langmap=Ant\:.xml:\
                            :langmap=Ant\:.ant:\
                            :langmap=Clojure\:.clj:\
                            :langmap=CoffeeScript\:.coffee:\
                            :langmap=C++\:.inl:\
                            :langmap=CSS\:.css:\
                            :langmap=ctags\:.ctags:\
                            :langmap=D\:.d.di:\
                            :langmap=Diff\:.diff.patch:\
                            :langmap=DTS\:.dts.dtsi:\
                            :langmap=Falcon\:.fal.ftd:\
                            :langmap=Fortran\:.f03.f08.f15:\
                            # gdbinit .gdbinit        (out of support)
                            :langmap=gdbinit\:.gdb:\
                                :langmap=Go\:.go:\
                                :langmap=JSON\:.json:\
                                :langmap=m4\:.m4.spt:\
                                :langmap=ObjectiveC\:.mm.m.h:\
                                :langmap=OCaml\:.aug:\
                                :langmap=Perl\:.ph:\
                                :langmap=Perl6\:.p6.pm6.pm.pl6:\
                                :langmap=PHP\:.php4.php5.php7:\
                                :langmap=R\:.r.R.s.q:\
                                :langmap=reStructuredText\:.rest.reST.rst:\
                                :langmap=Rust\:.rs:\
                                :langmap=Sh\:.ash:\
                                :langmap=SystemVerilog\:.sv.svh.svi:\
                                # Vim vimrc [._]vimrc gvimrc [._]gvimrc (out of support)
                                :langmap=Vim\:.vba:\
                                    :langmap=WindRes\:.rc:\
                                    :langmap=Zephir\:.zep:\
                                    # Please uncomment to use this entry.
                                    # :langmap=DBusIntrospect\:.xml:\
                                        # :langmap=Glade\:.glade:\
                                        :gtags_parser=Ada\:$ctagslib:\
                                        :gtags_parser=Ant\:$ctagslib:\
                                        :gtags_parser=Clojure\:$ctagslib:\
                                        :gtags_parser=CoffeeScript\:$ctagslib:\
                                        :gtags_parser=CSS\:$ctagslib:\
                                        :gtags_parser=ctags\:$ctagslib:\
                                        :gtags_parser=D\:$ctagslib:\
                                        :gtags_parser=Diff\:$ctagslib:\
                                        :gtags_parser=DTS\:$ctagslib:\
                                        :gtags_parser=Falcon\:$ctagslib:\
                                        :gtags_parser=gdbinit\:$ctagslib:\
                                        :gtags_parser=Go\:$ctagslib:\
                                        :gtags_parser=JSON\:$ctagslib:\
                                        :gtags_parser=m4\:$ctagslib:\
                                        :gtags_parser=ObjectiveC\:$ctagslib:\
                                        :gtags_parser=Perl6\:$ctagslib:\
                                        :gtags_parser=R\:$ctagslib:\
                                        :gtags_parser=reStructuredText\:$ctagslib:\
                                        :gtags_parser=Rust\:$ctagslib:\
                                        :gtags_parser=SystemVerilog\:$ctagslib:\
                                        :gtags_parser=WindRes\:$ctagslib:\
                                        :gtags_parser=Zephir\:$ctagslib:\
                                        :gtags_parser=DBusIntrospect\:$ctagslib:\
                                        :gtags_parser=Glade\:$ctagslib:
#
# Plug-in parser to use Pygments.
#
pygments-parser|Pygments plug-in parser:\
                         :ctagscom=/usr/bin/ctags:\
                         :pygmentslib=$libdir/gtags/pygments-parser.so:\
                         :tc=common:\
                         :langmap=ABAP\:.abap:\
                         :langmap=ANTLR\:.G.g:\
                         :langmap=ActionScript3\:.as:\
                         :langmap=Ada\:.adb.ads.ada:\
                         :langmap=AppleScript\:.applescript:\
                         :langmap=AspectJ\:.aj:\
                         :langmap=Aspx-cs\:.aspx.asax.ascx.ashx.asmx.axd:\
                         :langmap=Asymptote\:.asy:\
                         :langmap=AutoIt\:.au3:\
                         :langmap=Awk\:.awk.gawk.mawk:\
                         :langmap=BUGS\:.bug:\
                         :langmap=Bash\:.sh.ksh.bash.ebuild.eclass:\
                         :langmap=Bat\:.bat.cmd:\
                         :langmap=BlitzMax\:.bmx:\
                         :langmap=Boo\:.boo:\
                         :langmap=Bro\:.bro:\
                         :langmap=C#\:.cs:\
                         :langmap=C++\:.c++.cc.cp.cpp.cxx.h.h++.hh.hp.hpp.hxx.C.H:\
                         :langmap=COBOLFree\:.cbl.CBL:\
                         :langmap=COBOL\:.cob.COB.cpy.CPY:\
                         :langmap=CUDA\:.cu.cuh:\
                         :langmap=C\:.c.h:\
                         :langmap=Ceylon\:.ceylon:\
                         :langmap=Cfm\:.cfm.cfml.cfc:\
                         :langmap=Clojure\:.clj:\
                         :langmap=CoffeeScript\:.coffee:\
                         :langmap=Common-Lisp\:.cl.lisp.el:\
                         :langmap=Coq\:.v:\
                         :langmap=Croc\:.croc:\
                         :langmap=Csh\:.tcsh.csh:\
                         :langmap=Cython\:.pyx.pxd.pxi:\
                         :langmap=Dart\:.dart:\
                         :langmap=Dg\:.dg:\
                         :langmap=Duel\:.duel.jbst:\
                         :langmap=Dylan\:.dylan.dyl.intr:\
                         :langmap=ECL\:.ecl:\
                         :langmap=EC\:.ec.eh:\
                         :langmap=ERB\:.erb:\
                         :langmap=Elixir\:.ex.exs:\
                         :langmap=Erlang\:.erl.hrl.es.escript:\
                         :langmap=Evoque\:.evoque:\
                         :langmap=FSharp\:.fs.fsi:\
                         :langmap=Factor\:.factor:\
                         :langmap=Fancy\:.fy.fancypack:\
                         :langmap=Fantom\:.fan:\
                         :langmap=Felix\:.flx.flxh:\
                         :langmap=Fortran\:.f.f90.F.F90:\
                         :langmap=GAS\:.s.S:\
                         :langmap=GLSL\:.vert.frag.geo:\
                         :langmap=Genshi\:.kid:\
                         :langmap=Gherkin\:.feature:\
                         :langmap=Gnuplot\:.plot.plt:\
                         :langmap=Go\:.go:\
                         :langmap=GoodData-CL\:.gdc:\
                         :langmap=Gosu\:.gs.gsx.gsp.vark:\
                         :langmap=Groovy\:.groovy:\
                         :langmap=Gst\:.gst:\
                         :langmap=HaXe\:.hx:\
                         :langmap=Haml\:.haml:\
                         :langmap=Haskell\:.hs:\
                         :langmap=Hxml\:.hxml:\
                         :langmap=Hybris\:.hy.hyb:\
                         :langmap=IDL\:.pro:\
                         :langmap=Io\:.io:\
                         :langmap=Ioke\:.ik:\
                         :langmap=JAGS\:.jag.bug:\
                         :langmap=Jade\:.jade:\
                         :langmap=JavaScript\:.js:\
                         :langmap=Java\:.java:\
                         :langmap=Jsp\:.jsp:\
                         :langmap=Julia\:.jl:\
                         :langmap=Koka\:.kk.kki:\
                         :langmap=Kotlin\:.kt:\
                         :langmap=LLVM\:.ll:\
                         :langmap=Lasso\:.sosso:\
                         :langmap=Literate-Haskell\:.lhs:\
                         :langmap=LiveScript\:.ls:\
                         :langmap=Logos\:.x.xi.xm.xmi:\
                         :langmap=Logtalk\:.lgt:\
                         :langmap=Lua\:.lua.wlua:\
                         :langmap=MOOCode\:.moo:\
                         :langmap=MXML\:.mxml:\
                         :langmap=Mako\:.mao:\
                         :langmap=Mason\:.m.mhtml.mc.mi:\
                         :langmap=Matlab\:.m:\
                         :langmap=Modelica\:.mo:\
                         :langmap=Modula2\:.mod:\
                         :langmap=Monkey\:.monkey:\
                         :langmap=MoonScript\:.moon:\
                         :langmap=MuPAD\:.mu:\
                         :langmap=Myghty\:.myt:\
                         :langmap=NASM\:.asm.ASM:\
                         :langmap=NSIS\:.nsi.nsh:\
                         :langmap=Nemerle\:.n:\
                         :langmap=NewLisp\:.lsp.nl:\
                         :langmap=Newspeak\:.ns2:\
                         :langmap=Nimrod\:.nim.nimrod:\
                         :langmap=OCaml\:.ml.mli.mll.mly:\
                         :langmap=Objective-C++\:.mm.hh:\
                         :langmap=Objective-C\:.m.h:\
                         :langmap=Objective-J\:.j:\
                         :langmap=Octave\:.m:\
                         :langmap=Ooc\:.ooc:\
                         :langmap=Opa\:.opa:\
                         :langmap=OpenEdge\:.p.cls:\
                         :langmap=PHP\:.php.php3.phtml:\
                         :langmap=Pascal\:.pas:\
                         :langmap=Perl\:.pl.pm:\
                         :langmap=PostScript\:.ps.eps:\
                         :langmap=PowerShell\:.ps1:\
                         :langmap=Prolog\:.prolog.pro.pl:\
                         :langmap=Python\:.py.pyw.sc.tac.sage:\
                         :langmap=QML\:.qml:\
                         :langmap=REBOL\:.r.r3:\
                         :langmap=RHTML\:.rhtml:\
                         :langmap=Racket\:.rkt.rktl:\
                         :langmap=Ragel\:.rl:\
                         :langmap=Redcode\:.cw:\
                         :langmap=RobotFramework\:.robot:\
                         :langmap=Ruby\:.rb.rbw.rake.gemspec.rbx.duby:\
                         :langmap=Rust\:.rs.rc:\
                         :langmap=S\:.S.R:\
                         :langmap=Scala\:.scala:\
                         :langmap=Scaml\:.scaml:\
                         :langmap=Scheme\:.scm.ss:\
                         :langmap=Scilab\:.sci.sce.tst:\
                         :langmap=Smalltalk\:.st:\
                         :langmap=Smarty\:.tpl:\
                         :langmap=Sml\:.sml.sig.fun:\
                         :langmap=Snobol\:.snobol:\
                         :langmap=SourcePawn\:.sp:\
                         :langmap=Spitfire\:.spt:\
                         :langmap=Ssp\:.ssp:\
                         :langmap=Stan\:.stan:\
                         :langmap=SystemVerilog\:.sv.svh:\
                         :langmap=Tcl\:.tcl:\
                         :langmap=TeX\:.tex.aux.toc:\
                         :langmap=Tea\:.tea:\
                         :langmap=Treetop\:.treetop.tt:\
                         :langmap=TypeScript\:.ts:\
                         :langmap=UrbiScript\:.u:\
                         :langmap=VB.net\:.vb.bas:\
                         :langmap=VGL\:.rpf:\
                         :langmap=Vala\:.vala.vapi:\
                         :langmap=Velocity\:.vm.fhtml:\
                         :langmap=Verilog\:.v:\
                         :langmap=Vhdl\:.vhdl.vhd:\
                         :langmap=Vim\:.vim:\
                         :langmap=XBase\:.PRG.prg:\
                         :langmap=XQuery\:.xqy.xquery.xq.xql.xqm:\
                         :langmap=XSLT\:.xsl.xslt.xpl:\
                         :langmap=Xtend\:.xtend:\
                         :gtags_parser=ABAP\:$pygmentslib:\
                         :gtags_parser=ANTLR\:$pygmentslib:\
                         :gtags_parser=ActionScript3\:$pygmentslib:\
                         :gtags_parser=Ada\:$pygmentslib:\
                         :gtags_parser=AppleScript\:$pygmentslib:\
                         :gtags_parser=AspectJ\:$pygmentslib:\
                         :gtags_parser=Aspx-cs\:$pygmentslib:\
                         :gtags_parser=Asymptote\:$pygmentslib:\
                         :gtags_parser=AutoIt\:$pygmentslib:\
                         :gtags_parser=Awk\:$pygmentslib:\
                         :gtags_parser=BUGS\:$pygmentslib:\
                         :gtags_parser=Bash\:$pygmentslib:\
                         :gtags_parser=Bat\:$pygmentslib:\
                         :gtags_parser=BlitzMax\:$pygmentslib:\
                         :gtags_parser=Boo\:$pygmentslib:\
                         :gtags_parser=Bro\:$pygmentslib:\
                         :gtags_parser=C#\:$pygmentslib:\
                         :gtags_parser=C++\:$pygmentslib:\
                         :gtags_parser=COBOLFree\:$pygmentslib:\
                         :gtags_parser=COBOL\:$pygmentslib:\
                         :gtags_parser=CUDA\:$pygmentslib:\
                         :gtags_parser=C\:$pygmentslib:\
                         :gtags_parser=Ceylon\:$pygmentslib:\
                         :gtags_parser=Cfm\:$pygmentslib:\
                         :gtags_parser=Clojure\:$pygmentslib:\
                         :gtags_parser=CoffeeScript\:$pygmentslib:\
                         :gtags_parser=Common-Lisp\:$pygmentslib:\
                         :gtags_parser=Coq\:$pygmentslib:\
                         :gtags_parser=Croc\:$pygmentslib:\
                         :gtags_parser=Csh\:$pygmentslib:\
                         :gtags_parser=Cython\:$pygmentslib:\
                         :gtags_parser=Dart\:$pygmentslib:\
                         :gtags_parser=Dg\:$pygmentslib:\
                         :gtags_parser=Duel\:$pygmentslib:\
                         :gtags_parser=Dylan\:$pygmentslib:\
                         :gtags_parser=ECL\:$pygmentslib:\
                         :gtags_parser=EC\:$pygmentslib:\
                         :gtags_parser=ERB\:$pygmentslib:\
                         :gtags_parser=Elixir\:$pygmentslib:\
                         :gtags_parser=Erlang\:$pygmentslib:\
                         :gtags_parser=Evoque\:$pygmentslib:\
                         :gtags_parser=FSharp\:$pygmentslib:\
                         :gtags_parser=Factor\:$pygmentslib:\
                         :gtags_parser=Fancy\:$pygmentslib:\
                         :gtags_parser=Fantom\:$pygmentslib:\
                         :gtags_parser=Felix\:$pygmentslib:\
                         :gtags_parser=Fortran\:$pygmentslib:\
                         :gtags_parser=GAS\:$pygmentslib:\
                         :gtags_parser=GLSL\:$pygmentslib:\
                         :gtags_parser=Genshi\:$pygmentslib:\
                         :gtags_parser=Gherkin\:$pygmentslib:\
                         :gtags_parser=Gnuplot\:$pygmentslib:\
                         :gtags_parser=Go\:$pygmentslib:\
                         :gtags_parser=GoodData-CL\:$pygmentslib:\
                         :gtags_parser=Gosu\:$pygmentslib:\
                         :gtags_parser=Groovy\:$pygmentslib:\
                         :gtags_parser=Gst\:$pygmentslib:\
                         :gtags_parser=HaXe\:$pygmentslib:\
                         :gtags_parser=Haml\:$pygmentslib:\
                         :gtags_parser=Haskell\:$pygmentslib:\
                         :gtags_parser=Hxml\:$pygmentslib:\
                         :gtags_parser=Hybris\:$pygmentslib:\
                         :gtags_parser=IDL\:$pygmentslib:\
                         :gtags_parser=Io\:$pygmentslib:\
                         :gtags_parser=Ioke\:$pygmentslib:\
                         :gtags_parser=JAGS\:$pygmentslib:\
                         :gtags_parser=Jade\:$pygmentslib:\
                         :gtags_parser=JavaScript\:$pygmentslib:\
                         :gtags_parser=Java\:$pygmentslib:\
                         :gtags_parser=Jsp\:$pygmentslib:\
                         :gtags_parser=Julia\:$pygmentslib:\
                         :gtags_parser=Koka\:$pygmentslib:\
                         :gtags_parser=Kotlin\:$pygmentslib:\
                         :gtags_parser=LLVM\:$pygmentslib:\
                         :gtags_parser=Lasso\:$pygmentslib:\
                         :gtags_parser=Literate-Haskell\:$pygmentslib:\
                         :gtags_parser=LiveScript\:$pygmentslib:\
                         :gtags_parser=Logos\:$pygmentslib:\
                         :gtags_parser=Logtalk\:$pygmentslib:\
                         :gtags_parser=Lua\:$pygmentslib:\
                         :gtags_parser=MAQL\:$pygmentslib:\
                         :gtags_parser=MOOCode\:$pygmentslib:\
                         :gtags_parser=MXML\:$pygmentslib:\
                         :gtags_parser=Mako\:$pygmentslib:\
                         :gtags_parser=Mason\:$pygmentslib:\
                         :gtags_parser=Matlab\:$pygmentslib:\
                         :gtags_parser=MiniD\:$pygmentslib:\
                         :gtags_parser=Modelica\:$pygmentslib:\
                         :gtags_parser=Modula2\:$pygmentslib:\
                         :gtags_parser=Monkey\:$pygmentslib:\
                         :gtags_parser=MoonScript\:$pygmentslib:\
                         :gtags_parser=MuPAD\:$pygmentslib:\
                         :gtags_parser=Myghty\:$pygmentslib:\
                         :gtags_parser=NASM\:$pygmentslib:\
                         :gtags_parser=NSIS\:$pygmentslib:\
                         :gtags_parser=Nemerle\:$pygmentslib:\
                         :gtags_parser=NewLisp\:$pygmentslib:\
                         :gtags_parser=Newspeak\:$pygmentslib:\
                         :gtags_parser=Nimrod\:$pygmentslib:\
                         :gtags_parser=OCaml\:$pygmentslib:\
                         :gtags_parser=Objective-C++\:$pygmentslib:\
                         :gtags_parser=Objective-C\:$pygmentslib:\
                         :gtags_parser=Objective-J\:$pygmentslib:\
                         :gtags_parser=Octave\:$pygmentslib:\
                         :gtags_parser=Ooc\:$pygmentslib:\
                         :gtags_parser=Opa\:$pygmentslib:\
                         :gtags_parser=OpenEdge\:$pygmentslib:\
                         :gtags_parser=PHP\:$pygmentslib:\
                         :gtags_parser=Pascal\:$pygmentslib:\
                         :gtags_parser=Perl\:$pygmentslib:\
                         :gtags_parser=PostScript\:$pygmentslib:\
                         :gtags_parser=PowerShell\:$pygmentslib:\
                         :gtags_parser=Prolog\:$pygmentslib:\
                         :gtags_parser=Python\:$pygmentslib:\
                         :gtags_parser=QML\:$pygmentslib:\
                         :gtags_parser=REBOL\:$pygmentslib:\
                         :gtags_parser=RHTML\:$pygmentslib:\
                         :gtags_parser=Racket\:$pygmentslib:\
                         :gtags_parser=Ragel\:$pygmentslib:\
                         :gtags_parser=Redcode\:$pygmentslib:\
                         :gtags_parser=RobotFramework\:$pygmentslib:\
                         :gtags_parser=Ruby\:$pygmentslib:\
                         :gtags_parser=Rust\:$pygmentslib:\
                         :gtags_parser=S\:$pygmentslib:\
                         :gtags_parser=Scala\:$pygmentslib:\
                         :gtags_parser=Scaml\:$pygmentslib:\
                         :gtags_parser=Scheme\:$pygmentslib:\
                         :gtags_parser=Scilab\:$pygmentslib:\
                         :gtags_parser=Smalltalk\:$pygmentslib:\
                         :gtags_parser=Smarty\:$pygmentslib:\
                         :gtags_parser=Sml\:$pygmentslib:\
                         :gtags_parser=Snobol\:$pygmentslib:\
                         :gtags_parser=SourcePawn\:$pygmentslib:\
                         :gtags_parser=Spitfire\:$pygmentslib:\
                         :gtags_parser=Ssp\:$pygmentslib:\
                         :gtags_parser=Stan\:$pygmentslib:\
                         :gtags_parser=SystemVerilog\:$pygmentslib:\
                         :gtags_parser=Tcl\:$pygmentslib:\
                         :gtags_parser=TeX\:$pygmentslib:\
                         :gtags_parser=Tea\:$pygmentslib:\
                         :gtags_parser=Treetop\:$pygmentslib:\
                         :gtags_parser=TypeScript\:$pygmentslib:\
                         :gtags_parser=UrbiScript\:$pygmentslib:\
                         :gtags_parser=VB.net\:$pygmentslib:\
                         :gtags_parser=VGL\:$pygmentslib:\
                         :gtags_parser=Vala\:$pygmentslib:\
                         :gtags_parser=Velocity\:$pygmentslib:\
                         :gtags_parser=Verilog\:$pygmentslib:\
                         :gtags_parser=Vhdl\:$pygmentslib:\
                         :gtags_parser=Vim\:$pygmentslib:\
                         :gtags_parser=XBase\:$pygmentslib:\
                         :gtags_parser=XQuery\:$pygmentslib:\
                         :gtags_parser=XSLT\:$pygmentslib:\
                         :gtags_parser=Xtend\:$pygmentslib:
#
# Drupal configuration.
#
drupal|Drupal content management platform:\
              :tc=common:\
              :langmap=php\:.php.module.inc.profile.install.test:
#---------------------------------------------------------------------
# Configuration for htags(1)
#---------------------------------------------------------------------
htags:\
    ::
\end{lstlisting}

\section{\texttt{aria2}}
\label{sec:orgb284780}

\href{https://github.com/lilyvya/aria2-conf}{simplified-chinese}

\lstset{language=bash,label= ,caption= ,captionpos=b,numbers=none}
\begin{lstlisting}
# 更详细配置介绍请访问 https://aria2.github.io/manual/en/html/
# 全局代理
#all-proxy=http://127.0.0.1:1080/pac?auth=
# 代理请求方式,可用值 get, tunnel ,HTTPS 下载时一直使用 tunnel
#proxy-method=tunnel
# 不代理的主机名,域名和IP
#no-proxy=<DOMAINS>

# rpc
# 用户名
#rpc-user=user
# 密码
#rpc-passwd=passwd
# token验证
# 值可以为:我就是叫紫妈怎么了?有本事突然从我背后出现,把我的脸按在键盘上3sw4yde5uf6tgyhujikpo
rpc-secret=3sw4yde5uf6tgy7huj8ikp9o
# 允许rpc ,默认 false
enable-rpc=true
# 允许所有来源, web 界面跨域权限需要,默认 false
rpc-allow-origin-all=true
# 外部访问,默认 false
rpc-listen-all=true
# https加密,启用之后要设置公钥,私钥的文件路径
#rpc-secure=true
# 加密设置公钥
#rpc-certificate=example.crt
# 加密设置私钥
#rpc-private-key=example.key
# rpc端口,默认 6800
#rpc-listen-port=6800

# 下载
# 关闭 ipv6 ,默认 false
#disable-ipv6=true
# 最大同时下载数, 默认 5
#max-concurrent-downloads=3
# 断点续传,只在 HTTP(S) 和 FTP 中生效
continue=true
# HTTP 返回 503 时重试下载的秒数,0 为不重试下载,默认为 0
retry-wait=10
# 最大重试次数,0 为无限制,默认为 5
max-tries=0
#服务器返回文件找不到最大重试次数, 0 为一直重试,默认为 0
max-file-not-found=10
# 同服务器最大连接数,默认 1(好像不能使用过大的值,窝用20无法打开aria2)
max-connection-per-server=16
# 最小文件分段大小, 默认 20M
# 如果文件大小 < (min-split-size * split)则不分段
min-split-size=2M
# 单文件最大线程数, 默认 5
split=8
# 使用服务器文件时间,默认 false
remote-time=true
# 使用 UTF-8 编码,默认 false
content-disposition-default-utf8=true

# 如果相同的文件已存在重命名文件,默认 true
#auto-file-renaming=
# 总是重命名文件,默认 true
#always-resume=
# 最大能重命名几个文件,和 always-resume 有关
# 下载多个度盘打包文件的时候会重命名为 pack.zip, pack.1.zip... 这样,默认 0
max-resume-failure-tries=5

#referer ,这里由下载管理器调用
#如果为 * ,下载 URI 也可以是 referer ,这在开启 parameterized-uri 时很有用
#referer=<REFERER>
#参数化URI, balabala 不懂,默认 false
#Enable parameterized URI support. You can specify set of parts: http://{sv1,sv2,sv3}/foo.iso. Also you can specify numeric sequences with step counter: http://host/image[000-100:2].img. A step counter can be omitted. If all URIs do not point to the same file, such as the second example above, -Z option is required. Default: false
#parameterized-uri=true

# 全局下载速度限制
# 0为不受限制,默认为0,大小为 bytes/sec ,也可以使用 K 或 M,下同
#max-overall-download-limit=0
# 单文件下载速度限制
#max-download-limit=0
# 全局上传速度限制
#max-overall-upload-limit=100K
# 单文件上传速度限制
#max-upload-limit=0
# 断开速度过慢的连接,在 BT 中不生效
#lowest-speed-limit=0

# 文件保存路径,这里由下载管理器调用,默认为 Aria2 所在文件夹
#dir=
# 文件缓存,缓存到内存里,大小使用 K 或 M ,为 0 时关闭,默认 16M
disk-cache=64M
# 将文件映射到内存,如果文件没有预分配不能工作,默认 false
enable-mmap=true
# 映射内存文件大小最大限制
# 文件大小由一次下载中的全部文件决定,如果总大小大于设定值则禁用mmap
# 默认9223372036854775807 (单位有点迷)
max-mmap-limit=1280M
# 文件预分配
# 可用值 none, prealloc, trunc, falloc ,默认 prealloc
# 当使用新型文件系统,如 ext4, btrfs, xfs 或者 NTFS 文件系统时,推荐使用 falloc ,这种方式会在瞬间完成大文件(数 GB )的空间分配
# 不要在传统文件系统,如 ext3, FAT32 上使用 falloc ,因为这和使用 prealloc 所需的时间大同
# falloc 和 trunc 需要文件系统和内核支持
# 警告:使用 trunc 灰常快,但它其实是在文件系统中设置文件长度元数据,并不是分配磁盘空间,所以无法避免磁盘碎片化
# 简单来说开启文件预分配后,如果支持使用 falloc 就使用 falloc ,不支持就用 prealloc ,固态硬盘可以使用 trunc
# 开启后使用 32 位 aria2 下载大于 4G 的文件, aria2 会被系统杀掉 (骚年你还在用 32 位系统?)
# 使用 falloc 时,如果有警告[WARN] Gaining privilege SeManageVolumePrivilege failed. ,需要使用管理员权限打开
file-allocation=falloc
# 大小小于这个值的文件不进行文件预分配,默认 5M
no-file-allocation-limit=4096K
# 证书校验,默认 true
#check-certificate=false
# 证书效验文件
#ca-certificate=<FILE>

# 任务记录
#input-file=aria2.session
# 保存错误/未完成的任务记录
# 会在 dir 下生成同名 .aria2 文件,当 force-save=true 时不删除, force-save=false 时删除
#save-session=aria2.session
# 每隔几秒保存,默认 0, 0 为 aria2 退出时保存,不管设置为多少 aria2 退出时都会保存 session
save-session-interval=60
# 即使任务被移除或完成了也保存 session (使用 save-session 值) ,对 BT 有用,默认 false
# 会检查 session ,打开后如果按照本配置打开 aria2 下载下来的文件如果没有删除元数据会再给你下载下来
# 简单来说: false 不保存完成记录,true 保存完成记录
#force-save=true

#强迫症专用, force-save=true 时删除 dir 下同名.aria2
# .bat windows 用, .sh linux 用,需要在 bat/sh 里自己修改 dir
#on-download-complete=del.bat
#on-bt-download-complete=del_bt.bat
#on-download-complete=rm.sh
#on-bt-download-complete=rm_bt.bat

# 日志
# 级别,可用值 debug, info, notice, warn, error ,默认 debug
log-level=warn
# 位置
#log=aria2.log

# 轮询事件
# 可用值 epoll, kqueue, port, poll, select ,默认值与系统有关
# epoll, kqueue, port and poll 需要系统支持
# epoll 支持最新的 Linux , kqueue 支持最新的 BSD 系(包括 MAC OS X), port 支持 Open Solaris
#event-poll=

# BitTorrent
# 启用本地节点查找,默认 false,如果种子设置为 private , aria2 不会使用此选项
bt-enable-lpd=true
# 当值为 true| mem 时,如果下载的文件是一个种子(以.torrent结尾)时, 就自动下载
# 当值为 mem 时,种子不会写入磁盘,但会一直在内存中
# 当值为 false 时不自动下载
#  默认 true
#follow-torrent=flase
# 监听端口,默认6881-6999
# , 为分隔不同端口, - 为两个值中的所有端口,如:'6881-6889,6999', '6881,6885'
#listen-port=<PORT>...

# 强制加密,开启相当于 bt-require-crypto=true 和 bt-min-crypto-level=arc4 ,默认 false
bt-force-encryption=true
# 要求加密,默认 false
#bt-require-crypto=true
# 最低加密级别,可用值 plain, arc4 ,默认 plain
#bt-min-crypto-level=arc4

# 最大打开文件数量,默认 100
#bt-max-open-files=100
# 单种子最大连接数,默认 55 ,0 为不限制
#bt-max-peers=55
# 如果单种子的速度低于此值, aria2 会暂时增加种子的连接数量来增加下载速度,默认 50K
#bt-request-peer-speed-limit=50K

#添加额外的 tracker
#bt-tracker=<URI>[,...]
#要排除的 tracker,可以使用值 '*' (不要引号),当使用 * 时会移除所有通告的的 tracker
#bt-exclude-tracker=<URI>[,...]
# tracker 重连超时时间(单位为秒),默认 60
#bt-tracker-connect-timeout=60
#tracker 超时时间(单位为秒),默认 60
#bt-tracker-timeout=60

# DHT
# 打开 ipv4 DHT, 默认 true
# 当种子文件设置了 private ,即使为 true ,aria2也不会从 DHT 中下载, ipv6 也一样
#enable-dht=true
# 打开 ipv6 DHT,默认 true
#enable-dht6=true
# BT 和 DHT 使用的外部 IP ,可能会发送到BitTorrent tracker
# 对于DHT,此选项应该被设置成要报告的本地节点,这对于在 private 网络中使用 DHT 很重要
# For DHT, this option should be set to report that local node is downloading a particular torrent. This is critical to use DHT in a private network
#bt-external-ip=<IPADDRESS>
# DHT和 UDP tracker监听端口,默认 6881-6999
#dht-listen-port=<PORT>...
# 启用种子交换,默认 true ,如果种子设置为 private ,即使为 true 也不会启用这个特性
#enable-peer-exchange=true
# 做种流量比例, 0.0 时一直做种,默认 1.0
#seed-ratio=0.0

# 改变 IPv4 DHT 路由表保存路径,默认 $HOME/.aria2/dht.dat
#dht-file-path=dht.dat
# 改变 IPv6 DHT 路由表保存路径,默认 $HOME/.aria2/dht6.dat
#dht-file-path6=dht6.dat

# 修改 UA,默认 aria2/$VERSION
# 只在 HTTP(S) 中有效
#user-agent=<USER_AGENT>
# peer-id,默认 A2-$MAJOR-$MINOR-$PATCH- ,比如在 aria2 version 1.18.8 里为 A2-1-18-8-
# 在 BT 中只有前20个字符长度生效,超出的将被丢弃,不足则随机填充至20个字符
#peer-id-prefix=<PEER_ID_PREFIX>

# 保存元数据至 .torrent 文件,默认 false
#bt-save-metadata=true
# 从以前的种子下载不用验证散列,默认 false
# Seed previously downloaded files without verifying piece hashes
#bt-seed-unverified=true
# 种子哈希效验,开启时使用 check-integrity 选项和文件完成散列检查后才会继续,如果你想要下到损坏的文件那就关了吧。默认 true
# If true is given, after hash check using --check-integrity option and file is complete, continue to seed file. If you want to check file and download it only when it is damaged or incomplete, set this option to false. This option has effect only on BitTorrent download. Default: true
#bt-hash-check-seed=true
# 通过验证散列或整个文件的哈希值来检查文件完整性,只在 BT 中生效,HTTP(S)/FTP 使用 checksum 选项,默认 false
#check-integrity=false
\end{lstlisting}

\section{\texttt{Media}}
\label{sec:org94d266c}

\subsection{\texttt{mpv}}
\label{sec:org4e545cb}

\subsubsection{\texttt{config}}
\label{sec:org8ef30f0}

\lstset{language=bash,label= ,caption= ,captionpos=b,numbers=none}
\begin{lstlisting}
if [ -s /usr/share/doc/mpv/mpv.conf ]; then
    cp /usr/share/doc/mpv/mpv.conf ~/.config/mpv/mpv.conf
    echo "autosub-match=fuzzy" >> ~/.config/mpv/mpv.conf
fi
\end{lstlisting}

\lstset{language=bash,label= ,caption= ,captionpos=b,numbers=none}
\begin{lstlisting}
#
# Example mpv configuration file
#
# Warning:
#
# The commented example options usually do _not_ set the default values. Call
# mpv with --list-options to see the default values for most options. There is
# no builtin or example mpv.conf with all the defaults.
#
#
# Configuration files are read system-wide from /usr/local/etc/mpv.conf
# and per-user from ~/.config/mpv/mpv.conf, where per-user settings override
# system-wide settings, all of which are overridden by the command line.
#
# Configuration file settings and the command line options use the same
# underlying mechanisms. Most options can be put into the configuration file
# by dropping the preceding '--'. See the man page for a complete list of
# options.
#
# Lines starting with '#' are comments and are ignored.
#
# See the CONFIGURATION FILES section in the man page
# for a detailed description of the syntax.
#
# Profiles should be placed at the bottom of the configuration file to ensure
# that settings wanted as defaults are not restricted to specific profiles.

##################
# video settings #
##################

# Start in fullscreen mode by default.
#fs=yes

# force starting with centered window
#geometry=50%:50%

# don't allow a new window to have a size larger than 90% of the screen size
#autofit-larger=90%x90%

# Do not close the window on exit.
#keep-open=yes

# Do not wait with showing the video window until it has loaded. (This will
# resize the window once video is loaded. Also always shows a window with
# audio.)
#force-window=immediate

# Disable the On Screen Controller (OSC).
#osc=no

# Keep the player window on top of all other windows.
#ontop=yes

# Specify high quality video rendering preset (for OpenGL VO only)
# Can cause performance problems with some drivers and GPUs.
#profile=opengl-hq

# Force video to lock on the display's refresh rate, and change video and audio
# speed to some degree to ensure synchronous playback - can cause problems
# with some drivers and desktop environments.
#video-sync=display-resample

# Enable hardware decoding if available. Often, this does not work with all
# video outputs, but should work well with default settings on most systems.
# If performance or energy usage is an issue, forcing the vdpau or vaapi VOs
# may or may not help.
#hwdec=auto

##################
# audio settings #
##################

# Specify default audio device. You can list devices with: --audio-device=help
# The option takes the device string (the stuff between the '...').
#audio-device=alsa/default

# Do not filter audio to keep pitch when changing playback speed.
#audio-pitch-correction=no

# Output 5.1 audio natively, and upmix/downmix audio with a different format.
#audio-channels=5.1
# Disable any automatic remix, _if_ the audio output accepts the audio format.
# of the currently played file. See caveats mentioned in the manpage.
# (The default is "auto-safe", see manpage.)
#audio-channels=auto
autosub-match=fuzzy

##################
# other settings #
##################

# Pretend to be a web browser. Might fix playback with some streaming sites,
# but also will break with shoutcast streams.
#user-agent="Mozilla/5.0"

# cache settings
#
# Use 150MB input cache by default. The cache is enabled for network streams only.
#cache-default=153600
#
# Use 150MB input cache for everything, even local files.
#cache=153600
#
# Disable the behavior that the player will pause if the cache goes below a
# certain fill size.
#cache-pause=no
#
# Read ahead about 5 seconds of audio and video packets.
#demuxer-readahead-secs=5.0
#
# Raise readahead from demuxer-readahead-secs to this value if a cache is active.
#cache-secs=50.0

# Display English subtitles if available.
#slang=en

# Play Finnish audio if available, fall back to English otherwise.
#alang=fi,en

# Change subtitle encoding. For Arabic subtitles use 'cp1256'.
# If the file seems to be valid UTF-8, prefer UTF-8.
# (You can add '+' in front of the codepage to force it.)
#sub-codepage=cp1256

# You can also include other configuration files.
#include=/path/to/the/file/you/want/to/include

############
# Profiles #
############

# The options declared as part of profiles override global default settings,
# but only take effect when the profile is active.

# The following profile can be enabled on the command line with: --profile=eye-cancer

#[eye-cancer]
#sharpen=5
\end{lstlisting}

\subsubsection{\texttt{setting}}
\label{sec:org4319c6b}

\lstset{language=bash,label= ,caption= ,captionpos=b,numbers=none}
\begin{lstlisting}
seekbarstyle=bar
layout=bottombar
\end{lstlisting}

\lstset{language=bash,label= ,caption= ,captionpos=b,numbers=none}
\begin{lstlisting}
font=Sans Regular
font_size=12
alpha=70
duration=5
debug=yes
\end{lstlisting}

\subsubsection{\texttt{script}}
\label{sec:orgb018353}

\lstset{language=lua,label= ,caption= ,captionpos=b,numbers=none}
\begin{lstlisting}
-- This script automatically loads playlist entries before and after the
-- the currently played file. It does so by scanning the directory a file is
-- located in when starting playback. It sorts the directory entries
-- alphabetically, and adds entries before and after the current file to
-- the internal playlist. (It stops if the it would add an already existing
-- playlist entry at the same position - this makes it "stable".)
-- Add at most 5000 * 2 files when starting a file (before + after).
MAXENTRIES = 50

function Set (t)
  local set = {}
  for _, v in pairs(t) do set[v] = true end
  return set
end

EXTENSIONS = Set {
  'mkv', 'avi', 'mp4', 'ogv', 'webm', 'rmvb', 'flv', 'wmv', 'mpeg', 'mpg', 'm4v', '3gp',
  'mp3', 'wav', 'ogv', 'flac', 'm4a', 'wma',
}

mputils = require 'mp.utils'

function add_files_at(index, files)
  index = index - 1
  local oldcount = mp.get_property_number("playlist-count", 1)
  for i = 1, #files do
    mp.commandv("loadfile", files[i], "append")
    mp.commandv("playlist_move", oldcount + i - 1, index + i - 1)
  end
end

function get_extension(path)
  match = string.match(path, "%.([^%.]+)$" )
  if match == nil then
    return "nomatch"
  else
    return match
  end
end

table.filter = function(t, iter)
  for i = #t, 1, -1 do
    if not iter(t[i]) then
      table.remove(t, i)
    end
  end
end

function find_and_add_entries()
  local path = mp.get_property("path", "")
  local dir, filename = mputils.split_path(path)
  if #dir == 0 then
    return
  end
  local pl_count = mp.get_property_number("playlist-count", 1)
  if (pl_count > 1 and autoload == nil) or
  (pl_count == 1 and EXTENSIONS[string.lower(get_extension(filename))] == nil) then
    return
  else
    autoload = true
  end

  local files = mputils.readdir(dir, "files")
  if files == nil then
    return
  end
  table.filter(files, function (v, k)
                 local ext = get_extension(v)
                 if ext == nil then
                   return false
                 end
                 return EXTENSIONS[string.lower(ext)]
  end)
  table.sort(files, function (a, b)
               local len = string.len(a) - string.len(b)
               if len ~= 0 then -- case for ordering filename ending with such as X.Y.Z
                 local ext = string.len(get_extension(a)) + 1
                 return string.sub(a, 1, -ext) < string.sub(b, 1, -ext)
               end
               return string.lower(a) < string.lower(b)
  end)

  if dir == "." then
    dir = ""
  end

  local pl = mp.get_property_native("playlist", {})
  local pl_current = mp.get_property_number("playlist-pos", 0) + 1
  -- Find the current pl entry (dir+"/"+filename) in the sorted dir list
  local current
  for i = 1, #files do
    if files[i] == filename then
      current = i
      break
    end
  end
  if current == nil then
    return
  end

  local append = {[-1] = {}, [1] = {}}
  for direction = -1, 1, 2 do -- 2 iterations, with direction = -1 and +1
    for i = 1, MAXENTRIES do
      local file = files[current + i * direction]
      local pl_e = pl[pl_current + i * direction]
      if file == nil or file[1] == "." then
        break
      end

      local filepath = dir .. file
      if pl_e then
        -- If there's a playlist entry, and it's the same file, stop.
        if pl_e.filename == filepath then
          break
        end
      end

      if direction == -1 then
        if pl_current == 1 then -- never add additional entries in the middle
          mp.msg.info("Prepending " .. file)
          table.insert(append[-1], 1, filepath)
        end
      else
        mp.msg.info("Adding " .. file)
        table.insert(append[1], filepath)
      end
    end
  end

  add_files_at(pl_current + 1, append[1])
  add_files_at(pl_current, append[-1])
end

mp.register_event("start-file", find_and_add_entries)
\end{lstlisting}

\lstset{language=lua,label= ,caption= ,captionpos=b,numbers=none}
\begin{lstlisting}
-- Display some stats.
--
-- You can invoke the script with "i" by default or create a different key
-- binding in input.conf using "<yourkey> script_binding stats".
--
-- The style is configurable through a config file named "lua-settings/stats.conf"
-- located in your mpv directory.
--
-- Please note: not every property is always available and therefore not always
-- visible.

local options = require 'mp.options'

-- Options
local o = {
  -- Default key bindings
  key_oneshot = "i",
  key_toggle = "I",

  duration = 3,
  redraw_delay = 2,           -- acts as duration in the toggling case
  ass_formatting = true,
  debug = false,

  -- Text style
  font = "Source Sans Pro",
  font_size = 10,
  font_color = "FFFFFF",
  border_size = 1.0,
  border_color = "262626",
  shadow_x_offset = 0.0,
  shadow_y_offset = 0.0,
  shadow_color = "000000",
  alpha = "11",

  -- Custom header for ASS tags to style the text output.
  -- Specifying this will ignore the text style values above and just
  -- use this string instead.
  custom_header = "",

  -- Text formatting
  -- With ASS
  nl = "\\N",
  indent = "\\h\\h\\h\\h\\h",
  prefix_sep = "\\h\\h",
  b1 = "{\\b1}",
  b0 = "{\\b0}",
  -- Without ASS
  no_ass_nl = "\n",
  no_ass_indent = "\t",
  no_ass_prefix_sep = " ",
  no_ass_b1 = "\027[1m",
  no_ass_b0 = "\027[0m",
}
options.read_options(o)

function print_stats(duration)
  local stats = {
    header = "",
    file = "",
    video = "",
    audio = ""
  }

  o.ass_formatting = o.ass_formatting and has_vo_window()
  if not o.ass_formatting then
    o.nl = o.no_ass_nl
    o.indent = o.no_ass_indent
    o.prefix_sep = o.no_ass_prefix_sep
    if not has_ansi() then
      o.b1 = ""
      o.b0 = ""
    else
      o.b1 = o.no_ass_b1
      o.b0 = o.no_ass_b0
    end
  end

  add_header(stats)
  add_file(stats)
  add_video(stats)
  add_audio(stats)

  mp.osd_message(join_stats(stats), duration or o.duration)
end

function add_file(s)
  local sec = "file"
  s[sec] = ""

  append_property(s, sec, "filename", {prefix="File:", nl="", indent=""})
  if not (mp.get_property_osd("filename") == mp.get_property_osd("media-title")) then
    append_property(s, sec, "media-title", {prefix="Title:"})
  end
  append_property(s, sec, "chapter", {prefix="Chapter:"})
  if append_property(s, sec, "cache-used", {prefix="Cache:"}) then
    append_property(s, sec, "demuxer-cache-duration",
                    {prefix="+", suffix=" sec", nl="", indent=o.prefix_sep,
                     prefix_sep="", no_prefix_markup=true})
    append_property(s, sec, "cache-speed",
                    {prefix="", suffix="", nl="", indent=o.prefix_sep,
                     prefix_sep="", no_prefix_markup=true})
  end
end

function add_video(s)
  local sec = "video"
  s[sec] = ""
  if not has_video() then
    return
  end

  if append_property(s, sec, "video-codec", {prefix="Video:", nl="", indent=""}) then
    if not append_property(s, sec, "hwdec-current",
                           {prefix="(hwdec:", nl="", indent=" ",
                            no_prefix_markup=true, suffix=")"},
                           {no=true, [""]=true}) then
      append_property(s, sec, "hwdec-active",
                      {prefix="(hwdec)", nl="", indent=" ",
                       no_prefix_markup=true, no_value=true},
                      {no=true})
    end
  end
  append_property(s, sec, "avsync", {prefix="A-V:"})
  if append_property(s, sec, "drop-frame-count", {prefix="Dropped:"}) then
    append_property(s, sec, "vo-drop-frame-count", {prefix="VO:", nl=""})
    append_property(s, sec, "mistimed-frame-count", {prefix="Mistimed:", nl=""})
    append_property(s, sec, "vo-delayed-frame-count", {prefix="Delayed:", nl=""})
  end
  if append_property(s, sec, "display-fps", {prefix="Display FPS:", suffix=" (specified)"}) then
    append_property(s, sec, "estimated-display-fps",
                    {suffix=" (estimated)", nl="", indent=""})
  else
    append_property(s, sec, "estimated-display-fps",
                    {prefix="Display FPS:", suffix=" (estimated)"})
  end
  if append_property(s, sec, "fps", {prefix="FPS:", suffix=" (specified)"}) then
    append_property(s, sec, "estimated-vf-fps",
                    {suffix=" (estimated)", nl="", indent=""})
  else
    append_property(s, sec, "estimated-vf-fps",
                    {prefix="FPS:", suffix=" (estimated)"})
  end
  if append_property(s, sec, "video-speed-correction", {prefix="DS:"}) then
    append_property(s, sec, "audio-speed-correction",
                    {prefix="/", nl="", indent=" ", prefix_sep=" ", no_prefix_markup=true})
  end
  if append_property(s, sec, "video-params/w", {prefix="Native Resolution:"}) then
    append_property(s, sec, "video-params/h",
                    {prefix="x", nl="", indent=" ", prefix_sep=" ", no_prefix_markup=true})
  end
  append_property(s, sec, "window-scale", {prefix="Window Scale:"})
  append_property(s, sec, "video-params/aspect", {prefix="Aspect Ratio:"})
  append_property(s, sec, "video-params/pixelformat", {prefix="Pixel Format:"})
  append_property(s, sec, "video-params/colormatrix", {prefix="Colormatrix:"})
  append_property(s, sec, "video-params/primaries", {prefix="Primaries:"})
  append_property(s, sec, "video-params/gamma", {prefix="Gamma:"})
  append_property(s, sec, "video-params/colorlevels", {prefix="Levels:"})
  append_property(s, sec, "packet-video-bitrate", {prefix="Bitrate:", suffix=" kbps"})
end

function add_audio(s)
  local sec = "audio"
  s[sec] = ""
  if not has_audio() then
    return
  end

  append_property(s, sec, "audio-codec", {prefix="Audio:", nl="", indent=""})
  append_property(s, sec, "audio-params/samplerate", {prefix="Sample Rate:", suffix=" Hz"})
  append_property(s, sec, "audio-params/channel-count", {prefix="Channels:"})
  append_property(s, sec, "packet-audio-bitrate", {prefix="Bitrate:", suffix=" kbps"})
end

function add_header(s)
  if not o.ass_formatting then
    s.header = ""
    return
  end
  if o.custom_header and o.custom_header ~= "" then
    s.header = set_ASS(true) .. o.custom_header
  else
    s.header = string.format("%s{\\fs%d}{\\fn%s}{\\bord%f}{\\3c&H%s&}{\\1c&H%s&}" ..
                               "{\\alpha&H%s&}{\\xshad%f}{\\yshad%f}{\\4c&H%s&}",
                             set_ASS(true), o.font_size, o.font, o.border_size,
                             o.border_color, o.font_color, o.alpha, o.shadow_x_offset,
                             o.shadow_y_offset, o.shadow_color)
  end
end

-- Format and append a property.
-- A property whose value is either `nil` or empty (hereafter called "invalid")
-- is skipped and not appended.
-- Returns `false` in case nothing was appended, otherwise `true`.
--
-- s       : Table containing key `sec`.
-- sec     : Existing key in table `s`, value treated as a string.
-- property: The property to query and format (based on its OSD representation).
-- attr    : Optional table to overwrite certain (formatting) attributes for
--           this property.
-- exclude : Optional table containing keys which are considered invalid values
--           for this property. Specifying this will replace empty string as
--           default invalid value (nil is always invalid).
function append_property(s, sec, prop, attr, excluded)
  excluded = excluded or {[""] = true}
  local ret = mp.get_property_osd(prop)
  if not ret or excluded[ret] then
    if o.debug then
      print("No value for property: " .. prop)
    end
    return false
  end

  attr.prefix_sep = attr.prefix_sep or o.prefix_sep
  attr.indent = attr.indent or o.indent
  attr.nl = attr.nl or o.nl
  attr.suffix = attr.suffix or ""
  attr.prefix = attr.prefix or ""
  attr.no_prefix_markup = attr.no_prefix_markup or false
  attr.prefix = attr.no_prefix_markup and attr.prefix or b(attr.prefix)
  ret = attr.no_value and "" or ret

  s[sec] = string.format("%s%s%s%s%s%s%s", s[sec], attr.nl, attr.indent,
                         attr.prefix, attr.prefix_sep, no_ASS(ret), attr.suffix)
  return true
end

function no_ASS(t)
  return set_ASS(false) .. t .. set_ASS(true)
end

function set_ASS(b)
  if not o.ass_formatting then
    return ""
  end
  return mp.get_property_osd("osd-ass-cc/" .. (b and "0" or "1"))
end

function join_stats(s)
  r = s.header .. s.file

  if s.video and s.video ~= "" then
    r = r .. o.nl .. o.nl .. s.video
  end
  if s.audio and s.audio ~= "" then
    r = r .. o.nl .. o.nl .. s.audio
  end

  return r
end

function has_vo_window()
  return mp.get_property("vo-configured") == "yes"
end

function has_video()
  local r = mp.get_property("video")
  return r and r ~= "no" and r ~= ""
end

function has_audio()
  local r = mp.get_property("audio")
  return r and r ~= "no" and r ~= ""
end

function has_ansi()
  local is_windows = type(package) == 'table' and type(package.config) == 'string' and package.config:sub(1,1) == '\\'
  if is_windows then
    return os.getenv("ANSICON")
  end
  return true
end

function b(t)
  return o.b1 .. t .. o.b0
end

local timer = mp.add_periodic_timer(o.redraw_delay, function() print_stats(o.redraw_delay + 1) end)
timer:kill()

function toggle_stats()
  if timer:is_enabled() then
    timer:kill()
    mp.osd_message("", 0)
  else
    timer:resume()
    print_stats(o.redraw_delay + 1)
  end
end

mp.add_key_binding(o.key_oneshot, "display_stats", print_stats, {repeatable=true})
if pcall(function() timer:is_enabled() end) then
  mp.add_key_binding(o.key_toggle, "display_stats_toggle", toggle_stats, {repeatable=false})
else
  local txt = "Please upgrade mpv to toggle stats"
  mp.add_key_binding(o.key_toggle, "display_stats_toggle",
                     function() print(txt) ; mp.osd_message(txt) end, {repeatable=false})
end
\end{lstlisting}

\lstset{language=lua,label= ,caption= ,captionpos=b,numbers=none}
\begin{lstlisting}
-- Rebuild the terminal status line as a lua script
-- Be aware that this will require more cpu power!
-- Also, this is based on a rather old version of the
-- builtin mpv status line.

-- Add a string to the status line
function atsl(s)
  newStatus = newStatus .. s
end

function update_status_line()
  -- Reset the status line
  newStatus = ""

  if mp.get_property_bool("pause") then
    atsl("(Paused) ")
  elseif mp.get_property_bool("paused-for-cache") then
    atsl("(Buffering) ")
  end

  if mp.get_property("vid") ~= "no" then
    atsl("A")
  end
  if mp.get_property("aid") ~= "no" then
    atsl("V")
  end

  atsl(": ")

  atsl(mp.get_property_osd("time-pos"))

  atsl(" / ");
  atsl(mp.get_property_osd("duration"));

  atsl(" (")
  atsl(mp.get_property_osd("percent-pos", -1))
  atsl("%)")

  local r = mp.get_property_number("speed", -1)
  if r ~= 1 then
    atsl(string.format(" x%4.2f", r))
  end

  r = mp.get_property_number("avsync", nil)
  if r ~= nil then
    atsl(string.format(" A-V: %7.3f", r))
  end

  r = mp.get_property("total-avsync-change", 0)
  if math.abs(r) > 0.05 then
    atsl(string.format(" ct:%7.3f", r))
  end

  r = mp.get_property_number("drop-frame-count", -1)
  if r > 0 then
    atsl(" Late: ")
    atsl(r)
  end

  r = mp.get_property_number("cache", 0)
  if r > 0 then
    atsl(string.format(" Cache: %d%% ", r))
  end

  -- Set the new status line
  mp.set_property("options/term-status-msg", newStatus)
end

-- Register the event
mp.register_event("tick", update_status_line)
\end{lstlisting}
\end{document}